\documentclass[letterpaper, 10pt]{article}

\usepackage{luacode}

\usepackage[left=1in, top=1in, total={6.5in, 9in}]{geometry}

\usepackage{booktabs}

\usepackage{tabularx}

\usepackage{hyperref} 
\hypersetup{colorlinks, urlcolor=blue}
\urlstyle{same}

\usepackage{array}

\usepackage[]{ltablex}
\usepackage{makecell}
\renewcommand{\cellalign}{tl}

\setlength\LTleft{0pt}
\setlength\LTright{0pt}
\setlength{\LTpre}{0pt}
\setlength{\LTpost}{0pt}

\usepackage{titlesec}

\titleformat*{\section}{\large\bfseries}

\newcommand{\gutterSpacing}{0.125\textwidth}

\newcolumntype{D}{>{\itshape\arraybackslash}p{\gutterSpacing}}
\newcolumntype{R}{>{}p{\gutterSpacing}}

\usepackage{hyphenat}

\usepackage{lastpage}
\usepackage{fancyhdr}
\pagestyle{fancy}
\fancyhf{}
\lhead{Akshay Mehra}
\rhead{\textit{Curriculum Vitae, \thepage\ of \pageref*{LastPage}}}
\cfoot{}

\usepackage{siunitx}

% Custom command

\newcommand{\headings}[1]{\section*{#1} \hrule \vspace{10pt}}

\begin{document}

% Ensures that the first page doesn't have the header...
\thispagestyle{empty}

\begin{center}
	{\Large \textbf{Akshay Mehra}}\\
	Department of Earth Sciences \\
	225 Fairchild Hall \\
	Hanover, NH 03755 \\
	\href{mailto:akshay.k.mehra@dartmouth.edu}{akshay.k.mehra@dartmouth.edu} \\
	\href{https://www.akshaymehra.com}{https://www.akshaymehra.com}\\
\end{center}
	

\begin{luacode}
	-- My name (for citations)
	local myName = 'Mehra, A.'
	local myNameStylized = '\\textbf{Mehra, A.}'
	local f = io.open('cv.json', 'r')
	local s = f:read('*a')
	f:close()
	require("lualibs.lua")
	local data = utilities.json.tolua(s)
	-- A nice little function to test if a string is empty
	local function isEmpty(s)
		return s == nil or s == ''
	end
	-- A little function to see if an array contains a value
	local function matches(thisValue, arrayToMatch)
		for index, value in pairs(arrayToMatch) do
			if value == thisValue then
				return true
			end
		end
		return false
	end
	-- A little function to capitalize the first letter in a string
	function firstToUpper(str)
		return (str:gsub("^%l", string.upper))
	end
	-- Here, we prepare a string by replacing certain symbols with escape characters...this code is super hacky, but you can add to the escape list
	function prepareString(str)
		local escapeList = {"&"}
		for _, v in ipairs(escapeList) do
			local replaceWith = '\\'.. v 
			str = str:gsub(v, replaceWith)
		end
		return str
	end

	local prettyprint
	prettyprint = function (data)
		for key, value in pairs(data) do
			-- Okay, begin by printing out the headings
			tex.print('\\headings{' .. value["heading"] .. '}')
			-- Luacode has no switch statement, so we use a comprehensive if else chain
			-- What we're doing here is relying on a simple conceit: that certain sections have the same basic layout
			-- The most basic layout is last
			if matches(value["type"], {"publications"}) then
				-- Publications are challenging to lay out
				-- We begin by assembling tables
				-- Start by looking at the states of each publication
				local states = {}
				for dataKey, dataValue in ipairs(value["data"]) do
					table.insert(states, dataValue["state"])
				end
				-- Add each manuscript status you want to show, in the order that you want it to come out
				local orderedStates = {'in preparation', 'submitted', 'in review', 'published or in press'}
				local pubCount = #states
				for _, thisOrderedState in ipairs(orderedStates) do
					local needsHeading = true
					for pubIdx, pubState in ipairs(states) do
						-- If this publication state matches this ordered state, then let's print it out 
						if pubState == thisOrderedState then
							if needsHeading then
								tex.print('\\begin{flushleft}')
								tex.print(firstToUpper(pubState))
								tex.print('\\end{flushleft}')
								tex.print('\\begin{centering}')
								tex.print('\\renewcommand{\\arraystretch}{1.5}')
								tex.print('\\begin{tabularx}{\\textwidth}{@{}D X@{}}')
								needsHeading = false
							end
							local thisJournal = value["data"][pubIdx]["journal"]
							if isEmpty(thisJournal) then
								thisJournal = ''
							else
								thisJournal = ', \\textit{' .. thisJournal .. '}' 
							end
							local thisLink = value["data"][pubIdx]["link"]
							if isEmpty(thisLink) then
								thisLink = ''
							else
								thisLink = ', \\url{' .. thisLink .. '}'
							end
							local thisCitation = value["data"][pubIdx]["citation"] .. 
							thisJournal .. thisLink .. '.' 
							-- escape any problematic strings
							thisCitation =  prepareString(thisCitation)
							-- let's make my name bold
							thisCitation = thisCitation:gsub(myName,  myNameStylized)
							tex.print(pubCount .. '. & ' .. thisCitation .. '\\\\')
							pubCount = pubCount - 1
							if states[pubIdx + 1] ~= pubState then
								tex.print('\\end{tabularx}')
								tex.print('\\end{centering}')
							end
						end
					end
				end
			elseif matches(value["type"], {"funding", "teaching"}) then
				-- Calculate total funding
				local fundingTotal = 0
				if value["type"] == "funding" then
					for _, val in ipairs(value["data"]) do
						fundingTotal = fundingTotal + val["amount"]
					end
				end
				-- An endnote symbol
				local noteSymbol = '\\textsuperscript{\\dag}'
				-- A special three column layout
				tex.print('\\begin{center}') 
				tex.print('\\renewcommand{\\arraystretch}{1.5}')
				tex.print('\\begin{tabularx}{\\textwidth}{@{}D X R@{}}')
				for dataKey, dataValue in ipairs(value["data"]) do
					local columnOne = dataValue["date"]
					if isEmpty(columnOne) then
						columnOne = ''
					end
					local columnTwo = 'hi'
					local columnThree = ''
					if value["type"] == "funding" then
						local piListing = dataValue["pi"]
						piListing = piListing:gsub(myName,  myNameStylized)
						columnTwo = '\\makecell[{{p{\\hsize}}}]{\\textit{' .. dataValue["title"] .. '} \\\\' .. dataValue["source"] .. '\\\\' .. piListing .. '}'
						columnThree = '\\makecell[r]{\\SI[]{' .. dataValue["amount"] .. '}[\\$]{}}'
					elseif value["type"] == "teaching" then
						-- building this string iteratively
						columnTwo = '\\makecell[{{p{\\hsize}}}]{'
						if dataValue["hasEndnote"] then
							columnTwo = columnTwo .. noteSymbol
						end
						local courseTitle =  prepareString(dataValue["course_title"])
						columnTwo = columnTwo .. '\\textbf{' .. dataValue["course_number"] .. '}: ' .. courseTitle
						local fieldWork = dataValue["fieldwork"]
						if not isEmpty(fieldWork) then 
							fieldWork = '(' .. prepareString(fieldWork) .. ')'
							columnTwo = columnTwo .. ' \\\\ ' .. fieldWork  
						end
						columnTwo = columnTwo .. '}'
						columnThree = '\\makecell[{{p{\\hsize}}}]{\\raggedleft \\textit{' .. dataValue["role"] .. '}}'
					end
					tex.print(columnOne .. ' & ' .. columnTwo .. ' & ' .. columnThree .. ' \\\\' )
				end
				-- Optionally, print out the funding total and footnotes
				if value["type"] == "funding" then
					-- tex.print('& Total funding: & \\SI[]{' .. fundingTotal .. '}[\\$]{}')
				end 
				tex.print('\\end{tabularx}')
				tex.print('\\end{center}')
				-- Add any endnotes
				if value["type"] == "teaching" then
					local endnote = value["endnote"]
					if not isEmpty(endnote) then
						tex.print(noteSymbol .. '\\textit{' .. prepareString(endnote) .. '}')
					end
				end
			else
				tex.print('\\begin{center}') 
				tex.print('\\renewcommand{\\arraystretch}{1.5}')
				tex.print('\\begin{tabularx}{\\textwidth}{@{}D X@{}}')
				for dataKey, dataValue in ipairs(value["data"]) do
					local columnOne = dataValue["date"]
					if isEmpty(columnOne) then
						columnOne = ''
					end
					-- note, in some cases, you want column one not to be a date
					if value["type"] == "professionalActivities" then
						columnOne = dataValue["role"]
					end
					local thisContents
					if value["type"] == "education" then
						thisContents = '\\makecell[{{p{\\hsize}}}]{' .. dataValue["institution"] .. ', ' .. dataValue["location"] .. ' \\\\ ' .. dataValue["degree"] .. ' \\\\ Thesis: ' .. dataValue["thesis"] .. ' \\\\ Advisor(s): ' .. dataValue["advisor"] .. '}'
					elseif value["type"] == "professional" then
						thisContents = '\\makecell[{{p{\\hsize}}}]{' .. dataValue["role"] .. ' \\\\ ' .. dataValue["institution"] .. ', ' .. dataValue["location"] .. '}'
					elseif value["type"] == "conferences" then
						thisContents = dataValue["citation"]
						thisContents =  prepareString(thisContents)
						-- Make my name bold
						thisContents = thisContents:gsub(myName,  myNameStylized)
					elseif value["type"] == "fieldExperience" then
						thisContents = '\\makecell[{{p{\\hsize}}}]{' .. dataValue["location"] .. ', [' .. dataValue["length"] .. '] \\\\ \\textit{' .. dataValue["description"] .. '}}'
					else
						thisContents = dataValue["description"]
					end
					if isEmpty(thisContents) then
						thisContents = ''
					end
					tex.print(columnOne .. ' & ' .. thisContents .. ' \\\\')
				end
				tex.print('\\end{tabularx}')
				tex.print('\\end{center}')
			end
			-- One conceit to manual adjustment: add sections here that you want to force a page break after
			local sectionsToBreakAfter = {'conferences'}
			if matches(value["type"], sectionsToBreakAfter) then
				tex.print('\\pagebreak')
			end
		end
	end
	prettyprint(data)
\end{luacode}

\end{document}

